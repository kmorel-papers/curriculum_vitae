% -*- latex -*-

% This document uses the currvita environment.  For documentation, run
% latex on currvita.dtx (located in texmf/source/latex/currvita in MikTex)
% and view the resulting dvi.

\documentclass{article}

\usepackage{url}

% Resize the paper for smaller margins
\usepackage{geometry}
\geometry{letterpaper,tmargin=1in,bmargin=1in,lmargin=1in,rmargin=1in}

\setlength{\parskip}{\smallskipamount}
\setlength{\parindent}{0pt}

\usepackage[shortlabels]{enumitem}
\setitemize{noitemsep,topsep=0pt,parsep=0pt,partopsep=0pt}

\usepackage[resetlabels]{multibib}
\newcites{phd,journal,workshop,nopeer,poster}{%
  Ph.D. Thesis,%
  Journal and Conference Papers,%
  {Symposium, Workshop, and Invited Papers},%
  Technical Reports and other Non Peer Reviewed,%
  Posters%
}


% Change the references to use subsections
\usepackage{etoolbox}
\patchcmd{\thebibliography}{\section*{\refname}}{%
  \subsection*{\refname}
  \addcontentsline{toc}{subsection}{\refname}
}{}{}


%TIMESTAMP=`%B %e, %Y'
\date{March 23, 2023}

\sloppy

\begin{document}

\begin{center}
  \textbf{\LARGE Kenneth Dean Moreland}
\end{center}

\rule{\textwidth}{1mm}

\begin{center}
  \begin{tabular*}{\textwidth}{@{\extracolsep{\fill}}lr}
    Computer Science and Mathematics Division, Visualization  & \url{http://kennethmoreland.com/} \\
    Oak Ridge National Laboratory                 & morelandkd@ornl.gov \\
                                                  & (505) 440-6292 (cell) \\
  \end{tabular*}
\end{center}

\begin{center}
  {\large
    \makeatletter
    \@date
    \makeatother
  }
\end{center}


\section*{Education}

\begin{tabular}{l}
  Doctor of Philosophy, Computer Science, University of New Mexico, July 2004 \\
  Master of Science, Computer Science, University of New Mexico, May 2000 \\
  Bachelor of Science, Computer Science, New Mexico Institute of Mining and Technology, May 1997 \\
  Bachelor of Science, Electrical Engineering, New Mexico Institute of Mining and Technology, May 1997 \\
\end{tabular}


\section*{Employment}

\begin{description}
\item[Oak Ridge National Laboratory (Oak Ridge, TN)]\hfill 2021--Present\\
  2021-- Senior Research Scientist, Visualization, Computer Science and Mathematics Division

\item[Sandia National Laboratories (Albuquerque, NM)]\hfill 1995--2021\\
  2013--2021 Principal Member of Technical Staff, Scalable Analysis and Visualization\\
  2004--2013 Senior Member of Technical Staff, Scalable Analysis and Visualization\\
  2000--2004 Member of Technical Staff, Data Analysis and Visualization\\
  1999--2000 Member of Technical Staff, Distributed Systems Research\\
  1997--1999 Limited Term Employee, Computer Applications for Manufacturing\\
  1995--1997 Student Intern, Mechanical Process Engineering
\end{description}


\section*{Research Experience}
% Note: Funding for PI for total of project (where applicable) whereas co-PI just reports funding for our institution.

\begin{description}
\item[RAPIDS2]\hfill 2021--Present\\
  Work with DOE Office of Science application teams in addressing visualization challenges for science discovery.\\
  DOE Office of Science SciDAC Program\\
  Role: Key personnel
\item[ECP/VTK-m]\hfill 2017--Present\\
  Updating scientific visualization algorithms in the VTK-m framework for efficient visualization on multi- and many-core processor devices.\\
  DOE Office of Science Exascale Computing Project\\
  Role: PI\\
  Total funding: \$9.7M %(\$4.3M FY17-FY19, \$5.4M FY20-FY23)
\item[ATDM Scalable Visualization]\hfill 2016--2021\\
  Visualization support for mission-specific science applications for the Advanced Technology Development and Mitigation (ATDM) program.\\
  DOE NNSA ASC Program\\
  Role: PI\\
  Funding: \$800K/year
\item[XVis: Visualization for the Extreme-Scale Scientific-Computation Ecosystem]\hfill 2014--2017\\
  Providing the foundational research for visualization software for scientific discovery with exascale computing. This work was foundational to start software projects like \mbox{VTK-m} and provide improvements to in situ libraries like Catalyst. \\
  DOE Office of Science ASCR Program\\
  Role: PI \\
  Total funding: \$4M
\item[SciDAC Scalable Data Analysis and Visualization Institute]\hfill 2013--2017\\
  Provide technical solutions in the data management, analysis, and
  visualization regimes that are broadly applicable in the
  computational science community for the DOE Office of Science SciDAC program.\\
  DOE Office of Science SciDAC Program\\
  Role: Co-PI\\
  Funding: \$875K
\item[Data Analysis at Extreme (Dax)]\hfill 2010--2014\\
  Creating a framework for visualization on exascale processors. Dax was one of the precursors to \mbox{VTK-m}.\\
  DOE Office of Science ASCR Program\\
  Role: PI\\
  Total Funding: \$1.5M
\item[SciDAC Institute for Ultrascale Visualization]\hfill 2006--2012\\
  Addressed visualization challenges for the DOE Scientific Discovery through Advanced Computing effort.\\
  DOE Office of Science SciDAC Program\\
  Role: Co-PI\\
  Funding: \$900K
\item[ParaView Development Lead]\hfill 2006--2021\\
  Lead the ASC funded development effort for ParaView, a large-scale
  general visualization solution. Today, ParaView is downloaded hundreds of thousands of times each year.\\
  DOE NNSA ASC Program\\
  Role: PI\\
  Funding: \$1.6M/year % Taken from 7656/08.01 support budget for 2020
  %% \item[Scalable Visualization]\hfill 8/99--Present \\
  %%   Researched, developed, and deployed visualization algorithms designed
  %%   for scalability in parallel environments.
\item[Massive Graph Visualization]\hfill 2005--2007 \\
  Explored techniques for visualizing providing information in large
  graph structures.\\
  Sandia Laboratories LDRD Program\\
  Role: PI\\
  Funding: \$1.4M
  %% \item[Volume Rendering] Sandia National Laboratories, 8/03--8/07 \\
  %%   Researched and developed unstructured grid volume rendering codes for
  %%   rendering more accurate images in less time.  This includes utilizing
  %%   both CPU and GPU processors as well as employing parallel algorithms.
\item[Scalable Rendering]\hfill 1999--2005 \\
  Researched and developed parallel rendering codes targeted at performing scientific visualization on cluster computers.  Software targeted for very large inputs and/or to very large displays.\\
  DOE NNSA ASC Program\\
  Role: Key personnel
\item[Product Realization Environment]\hfill 1996--1999 \\
  Developed and deployed a CORBA-based middleware tool for distributing
  and connecting scientific modeling and simulation codes.\\
  DOE NNSA ASC Program\\
  Role: Key personnel
\end{description}


\section*{Publications}

% https://scholar.google.com/citations?user=vklmjtEAAAAJ&hl=en
Google Scholar Statistics [citations: 4362, h-index: 31, i10-index: 56]

\nocitephd{Moreland2004}
\bibliographystylephd{plainyr-rev}
\bibliographyphd{publications}

\nocitejournal{Moreland2021}
\nocitejournal{Moreland2020:SIAM}
\nocitejournal{Childs2020}
\nocitejournal{Choi2018}
\nocitejournal{Deelman2018}
\nocitejournal{Bauer2016}
\nocitejournal{Moreland2016:VTKm}
\nocitejournal{Moreland2016:VisView}
\nocitejournal{Moreland2015:ISC}
\nocitejournal{Oldfield2014}
\nocitejournal{Tchoua2013}
\nocitejournal{Childs2013}
\nocitejournal{Moreland2013:TVCG}
\nocitejournal{Moreland2011:SC}
\nocitejournal{Biddiscombe2007}
\nocitejournal{Wylie2001}
\bibliographystylejournal{plainyr-rev}
\bibliographyjournal{publications}

\nociteworkshop{Moreland2022:InSitu}
\nociteworkshop{Ayachit2021}
\nociteworkshop{Sane2021}
\nociteworkshop{Yenpure2019}
\nociteworkshop{Moreland2018}
\nociteworkshop{Lessley2017:Duplicate}
\nociteworkshop{Larsen2016}
\nociteworkshop{Moreland2016:HVEI}
\nociteworkshop{Ayachit2015}
\nociteworkshop{Moreland2015:SFI}
\nociteworkshop{Miller2014}
\nociteworkshop{Moreland2013:UltraVis}
\nociteworkshop{Moreland2012:PDAC}
\nociteworkshop{Moreland2012:LDAV}
\nociteworkshop{Moreland2011:PDAC}
\nociteworkshop{Moreland2011:LDAV}
\nociteworkshop{Fabian2011}
\nociteworkshop{Klasky2011}
\nociteworkshop{Moreland2009}
\nociteworkshop{Ma2009:SciDACReview}
\nociteworkshop{Ross2008}
\nociteworkshop{Ma2007}
\nociteworkshop{Moreland2007}
\nociteworkshop{Cedilnik2006}
\nociteworkshop{Moreland2004}
\nociteworkshop{Moreland2003}
\nociteworkshop{Moreland2003:FFT}
\nociteworkshop{Wylie2002}
\nociteworkshop{Moreland2001}
\bibliographystyleworkshop{plainyr-rev}
\bibliographyworkshop{publications}

\nocitenopeer{VTKmUsersGuide20}
\nocitenopeer{VTKmUsersGuide19}
\nocitenopeer{Moreland2022:NovelHardware}
\nocitenopeer{Moreland2022:DataReduceTR}
\nocitenopeer{VTKmUsersGuide18}
\nocitenopeer{VTKmUsersGuide17}
\nocitenopeer{VTKmUsersGuide16}
\nocitenopeer{Bennett2019}
\nocitenopeer{VTKmUsersGuide15}
\nocitenopeer{Moreland2019:XVis}
\nocitenopeer{VTKmUsersGuide14}
\nocitenopeer{VTKmUsersGuide13}
\nocitenopeer{Moreland2018:HPCManage}
\nocitenopeer{Bourgeois2018}
\nocitenopeer{ExascaleASCRReport}
\nocitenopeer{Workflows2015}
\nocitenopeer{Schroots2014}
\nocitenopeer{Moreland2014:DaxFinal}
\nocitenopeer{Rogers2013}
\nocitenopeer{Moreland2012:TR}
\nocitenopeer{Moreland2012:Ultravis}
\nocitenopeer{Sewell2012}
\nocitenopeer{Barrett2012}
\nocitenopeer{IceT}
\nocitenopeer{ScientificDiscoveryExascale2011}
\nocitenopeer{Moreland2010:TR}
\nocitenopeer{Ice2009}
\nocitenopeer{Thompson2009}
\nocitenopeer{Moreland2008:UltraVis}
\nocitenopeer{Karelitz2008}
\nocitenopeer{Moreland2008:CUG}
\nocitenopeer{Crossno2007}
\nocitenopeer{Moreland2007:TR}
\nocitenopeer{Karelitz2007}
\nocitenopeer{Moreland2006:KWS}
\nocitenopeer{ParaViewGuide}
\bibliographystylenopeer{plainyr-rev}
\bibliographynopeer{publications}

\nociteposter{Maynard2013:poster}
\nociteposter{Moreland2011:poster}
\nociteposter{Peterka2010:poster}
\nociteposter{Moreland2008:poster}
\bibliographystyleposter{plainyr-rev}
\bibliographyposter{publications}


\subsection*{Presentations}

\begin{enumerate}[label={[\arabic*]}]
\item DOE Visualization Tools and Capabilities.
  Kenneth Moreland and David Pugmire.
  PSI2 SciDAC All-Hands, February 22, 2023.
\item Performance Portability Pre-ECP and Post-ECP -- VTK-m.
  Kenneth Moreland.
  ECP Annual Meeting, January 19, 2023.
\item VTK-m -- A ToolKit for Scientific Visualization on Many-Core Processors.
  Tushar M. Athawale, Kenneth Moreland, David Pugmire, Silvio Rizzi, and Mark Bolstad.
  IEEE VIS, October 17, 2022.
\item Scientific Visualization on Supercomputers
  Kenneth Moreland.
  D\&AI Section All Hands Meeting, ORNL, September 21, 2022.
\item ParaView OLCF Tutorial.
  Kenneth Moreland.
  \emph{OLCF Training}, September 15, 2022.
\item VTK-m Update.
  Kenneth Moreland.
  \emph{DOE Computer Graphics Forum}, August 31, 2022.
\item ECP Data Management, Data Analytics, and Visualization Overview: VTK-m.
  ECP Annual Meeting, May 2, 2022.
\item Color: What it is and How to Use It to Show Data.
  Kenneth Moreland.
  VISTA Webinar, ORNL, September 9, 2021.
\item Enabling Visualization on Exascale Accelerators with VTK-m.
  Kenneth Moreland.
  RAPIDS2-DU Monthly Meeting, July 14, 2021.
\item Introduction to Scientific Visualization with ParaView 5.9.
  Kenneth Moreland.
  ORNL Software and Data Expo, May 20, 2021.
\item VTK-m Update.
  Kenneth Moreland.
  \emph{DOE Computer Graphics Forum}, April 28, 2021.
\item ECP Data Management, Data Analytics, and Visualization Overview: VTK-m.
  ECP Annual Meeting, March 30, 2021.
\item A Winding Road to Exascale Visualization.
  Kenneth Moreland.
  Oak Ridge National Laboratories, February 2, 2020.
\item What's New in ParaView.
  Kenneth Moreland.
  \emph{DOE Computer Graphics Forum}, April 28, 2020.
\item VTK-m Update.
  Kenneth Moreland.
  \emph{DOE Computer Graphics Forum}, April 28, 2020.
\item In Situ Visualization and Analysis with Ascent: Using VTK-m.
  Kenneth Moreland.
  \emph{ECP Annual Meeting} Tutorial, February 4, 2020.
\item VTK-m -- A ToolKit for Scientific Visualization on Many-Core Processors.
  Hank Childs, Kenneth Moreland, David Pugmire, Robert Maynard.
  \emph{IEEE VIS} Tutorial, October 2019.
\item Vis Capabilities at Sandia.
  Kenneth Moreland.
  Informal Review for 1540 Production Codes, June 10, 2019
\item What's New in ParaView.
  Kenneth Moreland.
  \emph{DOE Computer Graphics Forum}, April 23, 2019.
\item VTK-m Update.
  Kenneth Moreland.
  \emph{DOE Computer Graphics Forum}, April 23, 2019.
\item In Situ Visualization and Analysis with Ascent: Using VTK-m.
  Kenneth Moreland.
  ECP Annual Meeting Tutorial, January 17, 2019.
\item Building Better Plots.
  Kenneth Moreland.
  CCR Summer Seminar Talk, July 2018.
\item A Brief History of Interactive Visualization.
  Kenneth Moreland.
  University of Stuttgart Colloquium, June 29, 2018.
\item Preparations for Exascale Visualization at DOE.
  Kenneth Moreland.
  \emph{ISC High Performance}, June 25, 2018.
\item The Crazy Future of Vis.
  Kenneth Moreland, \emph{DOE Computer Graphics Forum}, Panel: 10-year Prognostication and How Do We Get There?, April 25, 2018.
\item What's New in ParaView.
  Kenneth Moreland, \emph{DOE Computer Graphics Forum}, April 24, 2018.
\item VTK-m Update.
  Kenneth Moreland, \emph{DOE Computer Graphics Forum}, April 24, 2018.
\item VTK-m: Visualization on Modern Processors.
  Kenneth Moreland, Kitware Booth, \emph{SC17}, November 15, 2017.
\item Large Scale Visualization with ParaView.
  Kenneth Moreland, W. Alan Scott, David E. DeMarle, Joe Insley, John Patchett, and Jon Woodring.
  Tutorial \emph{Supercomputing 2017}, November 14, 2017.
\item Why You Don't Want to do In Situ Visualization, and Why You Have To.
  Kenneth Moreland, \emph{Computational Science Seminar Series}, Sandia National Laboratories, October 31, 2017.
\item High Performance Visualization in the Many-Core Era.
  Kenneth Moreland, \emph{Computing@PNNL Seminar}, August 2017.
\item Why You Don't Want to do In Situ Visualization, and Why You Have To.
  Kenneth Moreland.
  \emph{ISC Workshop on In Situ Visualization}, June 2017.
\item The Many Faces and Solutions of In Situ Visualization.
  Kenneth Moreland.
  \emph{ISC Workshop on Visualization at Scale}, June 2017.
\item Making Better Plots.
  Kenneth Moreland.
  Sandia Post-Doc Workshop, February 20, 2017.
\item VTK-m: Building a Visualization Toolkit for Massively Threaded Architectures.
  Kenneth Moreland.
  \emph{Ultrascale Visualization Workshop}, November 17, 2015.
\item Large Scale Visualization with ParaView. Kenneth Moreland, W.
  Alan Scott. Tutorial \emph{International Conference on Supercomputing
    (ICS)}, November 2015.
\item Large Scale Visualization with ParaView. Kenneth Moreland, W.
  Alan Scott, Sabastien Jourdain, David DeMarle, Robert Maynard, Li-Ta
  Lo, and Joseph Insley. Tutorial \emph{Supercomputing 2014}, November
  2014.
\item Dax: A Massively Threaded Visualization and Analysis Toolkit for
  Extreme Scale. \emph{GPU Technology Conference}. March 2014.
\item Approaching Production In Situ Visualization for Extreme Scale
  Analysis. \emph{$16^{\mathrm{th}}$ SIAM Conference on Parallel
  Processing for Scientific Computing}. February 2014.
\item 15 Years of Large-Scale Scientific Visualization. University of
  Oregon colloquium. January 2014.
\item Large Scale Visualization with ParaView. Kenneth Moreland, W.
  Alan Scott, David DeMarle, and Li-Ta Lo. Tutorial
  \emph{Supercomputing 2013}, November 2013.
\item Large Scale Visualization with ParaView. Kenneth Moreland,
  W. Alan Scott, Nathan Fabian, Utkarsh Ayachit, and Robert
  Maynard. Tutorial \emph{Supercomputing 2012}, November 2012.
\item Next-Generation Capabilities for Large-Scale Scientific
  Visualization. Kenneth Moreland, Nathan Fabian, Berk Geveci, Utkarsh
  Ayachit, and James Ahrens. \emph{Massively Parallel, Scalable
    Algorithms and Softwares for Scientific Applications, 15th SIAM
    Conference on Parallel Processing for Scientific Computing},
  February 2012.
\item Flexible In Situ with ParaView. Kenneth Moreland, Nathan Fabian,
  Scott Klasky, and Berk Geveci. \emph{2011 Workshop on Ultrascale
    Visualization}, November 2011.
\item Large Scale Visualization with ParaView. Kenneth Moreland,
  W. Alan Scott, Nathan Fabian, Jeffrey Mauldin, Andrew C. Bauer,
  Robert Maynard, and Scott Klasky. Tutorial \emph{Supercomputing
    2011}, November 2011.
\item Large-Scale Interactive Visualization with ParaView. Kenneth
  Moreland. \emph{2nd International Conference on Computational
    Engineering}, October 2011.
\item Large Scale Visualization with ParaView. Kenneth
  Moreland. Tutorial \emph{SciDAC 2011}, July 2011.
\item In-Situ Visualization with the ParaView Coprocessing Library.
  Kenneth Moreland, Andrew Bauer, Pat Marion, and Nathan
  Fabian. Tutorial \emph{Supercomputing 2010}, November 2010.
\item Large Scale Visualization with ParaView. Kenneth Moreland,
  Andrew Bauer. Tutorial \emph{SciDAC 2010}, July 2010.
\item Large Scale Visualization with ParaView. Kenneth Moreland, John
  Greenfield, W. Alan Scott, Utkarsh Ayachit, and Berk Geveci. Tutorial
  \emph{Supercomputing 2009}, November 2009.
\item Parallel Distributed-Memory Visualization with ParaView. Kenenth
  Moreland and David DeMarle. Tutorial \emph{IEEE Cluster 2009}, August
  2009.
\item Large Scale Visualization with ParaView. Kenneth Moreland, John
  Greenfield, W. Alan Scott, Utkarsh Ayachit, Berk Geveci, and David
  DeMarle. Tutorial \emph{Supercomputing 2008}, November 2008.
\item Advanced ParaView Visualization. Kenneth Moreland, Utkarsh
  Ayachit, Timothy Shead, John Biddiscombe, and David
  Thompson. Tutorial \emph{IEEE Visualization 2008}, October 2008.
\item Large Scale Visualization with ParaView 3. Kenneth Moreland and
  John Greenfield. Tutorial Supercomputing 2007, November 2007.
\item Parallel Visualization with ParaView. Kenneth Moreland and Berk
  Geveci. Tutorial \emph{Supercomputing 2005}, November 2005.
\item Large Scale Visualization with Cluster Computing. Kenneth
  Moreland. \emph{Linux Cluster Institute Workshop}, October 2004.
\item Big Data, Big Displays, and Cluster-Driven Interactive
  Visualization. Kenneth Moreland. Colloquium, University of New
  Mexico, November 2002.
\item Big Data, Big Displays, and Cluster-Driven Interactive
  Visualization. Kenneth Moreland. \emph{Workshop on Commodity-Based
  Visualization Clusters}, October 2002.
\end{enumerate}


\subsection*{Panels}

\begin{enumerate}[label={[\arabic*]}]
\item Lessons Learned from Porting Codes to GPUs.
  ECP Annual Meeting, January 19, 2023.
\item Technical Advances in the Era of Big Data Analysis and Visualization: Large-Scale Visualization on Exascale Hardware.
  \emph{Large Data Analysis and Visualization (LDAV)}, October 25, 2020.
\item How Ubiquitous Parallel Devices Affect Visualization.
  \emph{Eurographics Symposium on Parallel Graphics and Visualization (EGPGV)}, 2020.
\item Color Mapping in Vis: Perspectives on Optimal Solutions.
  \emph{IEEE Visualization}, October 2015.
\end{enumerate}


\section*{Professional Activities}

\begin{description}
\item[Institute of Electrical and Electronic Engineers (IEEE)]
  Member since 1995.
\item[Association for Computing Machinery (ACM)]  Member since 1998.
\item[Editor]~
  \begin{itemize}
  \item
    Associate Editor, IEEE Transactions on Visualization and Computer Graphics (TVCG): 2022--Present.
  \item
    Guest Editor, \emph{Parallel Computing}: 2019.
  \end{itemize}
\item[Steering Committee Member]~
  \begin{itemize}
  \item
    IEEE Symposium on Large Data Analysis and Visualization (LDAV): 2020--Present.
  \item
    International Symposium on Visual Computing (ISVC): 2020--Present.
  \item
    In Situ Infrastructures for Enabling Extreme-scale Analysis and Visualization (ISAV): 2020--Present.
  \item
    Workshop on In Situ Visualization (WOIV): 2018--Present.
  \end{itemize}
\item[Event Chair/Co-Chair]~
  \begin{itemize}
  \item
    IEEE Symposium on Large Data Analysis and Visualization (LDAV): 2018, 2019.
  \item
    In Situ Infrastructures for Enabling Extreme-scale Analysis and Visualization (ISAV): 2019.
  \item
    Scientific Visualization \& Data Analytics Showcase, Supercomputing: 2018.
  \item
    VisLies: 2013, 2015, 2016, 2017, 2018, 2019, 2020, 2021, 2022.
  \end{itemize}
\item[Program Chair/Co-Chair]~
  \begin{itemize}
  \item
    In Situ Infrastructures for Enabling Extreme-scale Analysis and Visualization (ISAV): 2017, 2018.
  \item
    IEEE Symposium on Large Data Analysis and Visualization (LDAV): 2016, 2017.
  \item
    Eurographics Symposium on Parallel Graphics and Visualization (EGPGV): 2013.
  \end{itemize}
\item[Program Committee Member]~
  \begin{itemize}
  \item IEEE VIS: 2023
  \item
    EuroVis: 2016, 2017, 2018, 2021, 2022, 2023.
  \item
    Scientific Visualization \& Data Analytics Showcase, Supercomputing: 2017, 2023.
  \item
    International Conference on Information Visualization Theory and Applications (IVAPP): 2022, 2023. 
  \item
    International Symposium on Visual Computing (ISVC): 2013, 2014, 2019, 2020, 2021, 2022, 2023.
  \item
    IEEE VIS Short Papers: 2022.
  \item
    IEEE Scientific Visualization: 2014, 2015, 2016, 2017, 2019, 2020, 2021.
  \item
    Eurographics Symposium on Parallel Graphics and Visualization (EGPGV): 2009, 2010, 2011, 2012, 2013, 2015, 2016, 2017, 2019, 2020, 2021.
  \item
    IEEE Cluster: 2015, 2016, 2021.
  \item
    IEEE TPDS special section on Innovative R\&D toward the Exascale Era, 2021.
  \item
    VisGap: 2020.
  \item
    Symposium on Visualization: 2017.
  \item
    Visualization in High Performance Computing: 2015.
  \item
    Large Data Analysis and Visualization (LDAV): 2011, 2013, 2014.
  \item
    Ultrascale Visualization: 2013, 2014.
  \item
    International Conference for High Performance Computing, Networking, Storage, and Analysis (Supercomputing): 2012, 2013.
  \item
    Big Data Analytics: Challenges and Opportunities: 2012, 2013.
  \item
    Visualization Infrastructure \& Systems Technology: 2014.
  \end{itemize}
\item[Review Panels]~
  \begin{itemize}
  \item
    NSF IIS Division: 2012, 2014, 2015, 2016, 2018.
  \item
    DOE SBIR Phase II: 2012, 2014.
  \item
    DOE ECRP: 2009, 2013.
  \item
    DOE ASCR: 2009, 2010, 2011, 2013.
  \item
    DOE SBIR Phase I: 2010, 2011.
  \end{itemize}
\item[Miscellaneous]~
  \begin{itemize}
  \item
    Best Reviewer Award, EuroVis 2022.
  \item
    VIS Doctoral Colloquium Panelist: 2017.
  \end{itemize}
\end{description}

\section*{Software Artifacts}

The following is selected software I contributed to over (either as a developer or a project manager) the course of my career.

\begin{description}
\item[ParaView] A parallel scientific visualization application.
\item[Catalyst] An \emph{in situ} visualization library built on top of the ParaView application.
\item[VTK] A visualization library on which many tools, including ParaView, are built on top of.
\item[VTK-m] A library of visualization algorithms designed to run on multi- and many-core processors such as GPUs.
\item[IceT] A parallel rendering library.
\end{description}

\end{document}
