% -*- latex -*-

% This document uses the currvita environment.  For documentation, run
% latex on currvita.dtx (located in texmf/source/latex/currvita in MikTex)
% and view the resulting dvi.

\documentclass{article}

\usepackage{url}

% Resize the paper for smaller margins
\usepackage{geometry}
\geometry{letterpaper,tmargin=1in,bmargin=1in,lmargin=1in,rmargin=1in}

\setlength{\parskip}{\smallskipamount}
\setlength{\parindent}{0pt}

\usepackage[shortlabels]{enumitem}
\setitemize{noitemsep,topsep=0pt,parsep=0pt,partopsep=0pt}


% Change the references to use subsections
\usepackage{etoolbox}
\patchcmd{\thebibliography}{\section*{\refname}}{%
  \subsection*{\refname}
  \addcontentsline{toc}{subsection}{\refname}
}{}{}

\hyphenation{Para-View}


\date{\today}

\sloppy

\begin{document}

\begin{center}
  \textbf{\LARGE Kenneth Dean Moreland}
\end{center}

\rule{\textwidth}{1mm}

\begin{center}
  \begin{tabular*}{\textwidth}{@{\extracolsep{\fill}}lr}
    Computer Science and Mathematics Division, Visualization  & \url{https://www.kennethmoreland.com/} \\
    Oak Ridge National Laboratory                 & morelandkd@ornl.gov \\
                                                  & (505) 440-6292 (cell) \\
  \end{tabular*}
\end{center}

\begin{center}
  {\large
    \makeatletter
    \@date
    \makeatother
  }
\end{center}


\section*{Education}

\begin{tabular}{l}
  Doctor of Philosophy, Computer Science, University of New Mexico, July 2004 \\
  Master of Science, Computer Science, University of New Mexico, May 2000 \\
  Bachelor of Science, Computer Science, New Mexico Institute of Mining and Technology, May 1997 \\
  Bachelor of Science, Electrical Engineering, New Mexico Institute of Mining and Technology, May 1997 \\
\end{tabular}


\section*{Employment}

\begin{description}
\item[Oak Ridge National Laboratory (Oak Ridge, TN)]\hfill 2021--Present\\
  2021-- Senior Research Scientist, Visualization, Computer Science and Mathematics Division

\item[Sandia National Laboratories (Albuquerque, NM)]\hfill 1995--2021\\
  2013--2021 Principal Member of Technical Staff, Scalable Analysis and Visualization\\
  2004--2013 Senior Member of Technical Staff, Scalable Analysis and Visualization\\
  2000--2004 Member of Technical Staff, Data Analysis and Visualization\\
  1999--2000 Member of Technical Staff, Distributed Systems Research\\
  1997--1999 Limited Term Employee, Computer Applications for Manufacturing\\
  1995--1997 Student Intern, Mechanical Process Engineering
\end{description}


\section*{Research Experience}
% Note: Funding for PI for total of project (where applicable) whereas co-PI just reports funding for our institution.

\begin{description}
\item[RAPIDS2]\hfill 2021--Present\\
  Work with DOE Office of Science application teams in addressing visualization challenges for science discovery.\\
  DOE Office of Science SciDAC Program\\
  Role: Key personnel
\item[ECP/VTK-m]\hfill 2017--2023\\
  Updating scientific visualization algorithms in the VTK-m framework for efficient visualization on multi- and many-core processor devices.\\
  DOE Office of Science Exascale Computing Project\\
  Role: PI\\
  Total funding: \$9.7M %(\$4.3M FY17-FY19, \$5.4M FY20-FY23)
\item[ATDM Scalable Visualization]\hfill 2016--2021\\
  Visualization support for mission-specific science applications for the Advanced Technology Development and Mitigation (ATDM) program.\\
  DOE NNSA ASC Program\\
  Role: PI\\
  Funding: \$800K/year
\item[XVis: Visualization for the Extreme-Scale Scientific-Computation Ecosystem]\hfill 2014--2017\\
  Providing the foundational research for visualization software for scientific discovery with exascale computing. This work was foundational to start software projects like \mbox{VTK-m} and provide improvements to in situ libraries like Catalyst. \\
  DOE Office of Science ASCR Program\\
  Role: PI \\
  Total funding: \$4M
\item[SciDAC Scalable Data Analysis and Visualization Institute]\hfill 2013--2017\\
  Provide technical solutions in the data management, analysis, and
  visualization regimes that are broadly applicable in the
  computational science community for the DOE Office of Science SciDAC program.\\
  DOE Office of Science SciDAC Program\\
  Role: Co-PI\\
  Funding: \$875K
\item[Data Analysis at Extreme (Dax)]\hfill 2010--2014\\
  Creating a framework for visualization on exascale processors. Dax was one of the precursors to \mbox{VTK-m}.\\
  DOE Office of Science ASCR Program\\
  Role: PI\\
  Total Funding: \$1.5M
\item[SciDAC Institute for Ultrascale Visualization]\hfill 2006--2012\\
  Addressed visualization challenges for the DOE Scientific Discovery through Advanced Computing effort.\\
  DOE Office of Science SciDAC Program\\
  Role: Co-PI\\
  Funding: \$900K
\item[ParaView Development Lead]\hfill 2006--2021\\
  Lead the ASC funded development effort for ParaView, a large-scale
  general visualization solution. Today, ParaView is downloaded hundreds of thousands of times each year.\\
  DOE NNSA ASC Program\\
  Role: PI\\
  Funding: \$1.6M/year % Taken from 7656/08.01 support budget for 2020
  %% \item[Scalable Visualization]\hfill 8/99--Present \\
  %%   Researched, developed, and deployed visualization algorithms designed
  %%   for scalability in parallel environments.
\item[Massive Graph Visualization]\hfill 2005--2007 \\
  Explored techniques for visualizing providing information in large
  graph structures.\\
  Sandia Laboratories LDRD Program\\
  Role: PI\\
  Funding: \$1.4M
  %% \item[Volume Rendering] Sandia National Laboratories, 8/03--8/07 \\
  %%   Researched and developed unstructured grid volume rendering codes for
  %%   rendering more accurate images in less time.  This includes utilizing
  %%   both CPU and GPU processors as well as employing parallel algorithms.
\item[Scalable Rendering]\hfill 1999--2005 \\
  Researched and developed parallel rendering codes targeted at performing scientific visualization on cluster computers.  Software targeted for very large inputs and/or to very large displays.\\
  DOE NNSA ASC Program\\
  Role: Key personnel
\item[Product Realization Environment]\hfill 1996--1999 \\
  Developed and deployed a CORBA-based middleware tool for distributing
  and connecting scientific modeling and simulation codes.\\
  DOE NNSA ASC Program\\
  Role: Key personnel
\end{description}


\section*{Publications}

% https://scholar.google.com/citations?hl=en&user=d00I5gUAAAAJ
Google Scholar Statistics [citations: 5496, h-index: 33, i10-index: 62]

\subsection*{Ph.D. Thesis}
% DO NOT EDIT THIS FILE!
% This content is automatically generated from phd.bib.

\begin{enumerate}[label={[\arabic*]}, left=0pt]
\item  % Moreland2004
  \textbf{Kenneth Moreland}.
  \emph{Fast High Accuracy Volume Rendering}.
PhD thesis, University of New Mexico, July 2004.
\end{enumerate}

\subsection*{Journal and Conference Papers}
\begin{enumerate}[label={[\arabic*]}, left=0pt]
\item  % Wang2025:participation
  Zhe Wang, \textbf{Kenneth Moreland}, Matthew Larsen, James Kress, Hank Childs, and David Pugmire.
  Parallelize Over Data Particle Advection: Participation, Ping Pong Particles, and Overhead.
  \emph{IEEE Transactions on Visualization and Computer Graphics}, 10, 31(10):7795--7808, October 2025.
  doi:10.1109/TVCG.2025.3557453.
\item  % Wang2025:assignment
  Zhe Wang, \textbf{Kenneth Moreland}, Matthew Larsen, James Kress, Hank Childs, Guan Li, Guihua Shan, and David Pugmire.
  {\it In Situ} Workload Estimation for Block Assignment and Duplication in Parallelization-Over-Data Particle Advection.
  \emph{Computer Graphics Forum}, 3, 44(3), May 2025.
  doi:10.1111/cgf.70108.
\item  % Athawale2025
  Tushar M. Athawale, Zhe Wang, David Pugmire, \textbf{Kenneth Moreland}, Qian Gong, Scott Klasky, Chris R. Johnson, and Paul Rosen.
  Uncertainty Visualization of Critical Points of {2D} Scalar Fields for Parametric and Nonparametric Probabilistic Models.
  \emph{IEEE Transactions on Visualization and Computer Graphics}, 1, 31(1):108--118, January 2025.
  doi:10.1109/TVCG.2024.3456393.
\item  % Moreland2024:IJHPCA
  \textbf{Kenneth Moreland}, Tushar M. Athawale, Vicente Bolea, Mark Bolstad, Eric Brugger, Hank Childs, Axel Huebl, Li-Ta Lo, Berk Geveci, Nicole Marsaglia, Sujin Philip, David Pugmire, Silvio Rizzi, Zhe Wang, and Abhishek Yenpure.
  {Visualization at exascale: Making it all work with VTK-m}.
  \emph{The International Journal of High Performance Computing Applications}, 5, 38(5):508--526, August 2024.
  doi:10.1177/10943420241270969.
\item  % Samsel2024
  Francesca Samsel, W. Alan Scott, and \textbf{Kenneth Moreland}.
  A New Default Colormap for {ParaView}.
  \emph{IEEE Computer Graphics and Applications}, 4, 44(4):150--160, July 2024.
  doi:10.1109/MCG.2024.3383137.
\item  % Moreland2021
  \textbf{Kenneth Moreland}, Robert Maynard, David Pugmire, Abhishek Yenpure, Allison Vacanti, Matthew Larsen, and Hank Childs.
  Minimizing Development Costs for Efficient Many-Core Visualization Using {MCD$^3$}.
  \emph{Parallel Computing}, 102834, 108(102834), December 2021.
  doi:10.1016/j.parco.2021.102834.
\item  % Moreland2020:SIAM
  \textbf{Kenneth Moreland} and Hank Childs.
  Scientific Visualization: New Techniques in Production Software.
  \emph{SIAM News}, November 2020.
\item  % Childs2020
  Hank Childs, \emph{et al}.
  A terminology for in situ visualization and analysis systems.
  \emph{The International Journal of High Performance Computing Applications}, August 2020.
  doi:10.1177/1094342020935991.
\item  % Choi2018
  J. Y. Choi, C. Chang, J. Dominski, S. Klasky, G. Merlo, E. Suchyta, M. Ainsworth, B. Allen, F. Cappello, M. Churchill, P. Davis, S. Di, G. Eisenhauer, S. Ethier, I. Foster, B. Geveci, H. Guo, K. Huck, F. Jenko, M. Kim, J. Kress, S. Ku, Q. Liu, J. Logan, A. Malony, K. Mehta, \textbf{K. Moreland}, T. Munson, M. Parashar, T. Peterka, N. Podhorszki, D. Pugmire, O. Tugluk, R. Wang, B. Whitney, M. Wolf, and C. Wood.
  Coupling Exascale Multiphysics Applications: Methods and Lessons Learned.
  In \emph{2018 IEEE 14th International Conference on e-Science (e-Science)}, pages 442--452, October 2018.
  doi:10.1109/eScience.2018.00133.
\item  % Deelman2018
  Ewa Deelman, Tom Peterka, Ilkay Altintas, Christopher D Carothers, Kerstin Kleese van Dam, \textbf{Kenneth Moreland}, Manish Parashar, Lavanya Ramakrishnan, Michela Taufer, and Jeffrey Vetter.
  The future of scientific workflows.
  \emph{International Journal of High Performance Computing Applications}, 1, 32(1):159--175, January 2018.
  doi:10.1177/1094342017704893.
\item  % Bauer2016
  Andrew C. Bauer, Hasan Abbasi, James Ahrens, Hank Childs, Berk Geveci, Scott Klasky, \textbf{Kenneth Moreland}, Patrick O'Leary, Venkatram Vishwanath, Brad Whitlock, and E. Wes Bethel.
  In Situ Methods, Infrastructures, and Applications on High Performance Computing Platforms.
  \emph{Computer Graphics Forum}, 3, 35(3):577--597, June 2016.
  doi:10.1111/cgf.12930.
\item  % Moreland2016:VTKm
  \textbf{Kenneth Moreland}, Christopher Sewell, William Usher, Li-Ta Lo, Jeremy Meredith, David Pugmire, James Kress, Hendrik Schroots, Kwan-Liu Ma, Hank Childs, Matthew Larsen, Chun-Ming Chen, Robert Maynard, and Berk Geveci.
  {VTK-m}: Accelerating the Visualization Toolkit for Massively Threaded Architectures.
  \emph{IEEE Computer Graphics and Applications}, 3, 36(3):48--58, May/June 2016.
  doi:10.1109/MCG.2016.48.
\item  % Moreland2016:VisView
  \textbf{Kenneth Moreland}.
  The Tensions of In Situ Visualization.
  \emph{IEEE Computer Graphics and Applications}, 2, 36(2):5--9, March/April 2016.
  doi:10.1109/MCG.2016.35.
\item  % Moreland2015:ISC
  \textbf{Kenneth Moreland} and Ron Oldfield.
  Formal Metrics for Large-Scale Parallel Performance.
  In \emph{ISC High Performance}, pages 488--496, June 2015.
  doi:10.1007/978-3-319-20119-1\_34.
\item  % Oldfield2014
  Ron A. Oldfield, \textbf{Kenneth Moreland}, Nathan Fabian, and David Rogers.
  Evaluation of Methods to Integrate Analysis into a Large-Scale Shock Physics Code.
  In \emph{Proceedings of the 28th ACM international Conference on Supercomputing (ICS '14)}, pages 83--92, June 2014.
  doi:10.1145/2597652.2597668.
\item  % Tchoua2013
  Roselyne Tchoua, Jong Choi, Scott Klasky, Qing Liu, Jeremy Logan, \textbf{Kenneth Moreland}, Jingqing Mu, Manish Parashar, Norbert Podhorszki, David Pugmire, and Matthew Wolf.
  {ADIOS} Visualization Schema: A First Step Towards Improving Interdisciplinary Collaboration in High Performance Computing.
  In \emph{IEEE International Conference on eScience}, pages 27--34, October 2013.
  doi:10.1109/eScience.2013.24.
\item  % Childs2013
  Hank Childs, Berk Geveci, Will Schroeder, Jeremy Meredith, \textbf{Kenneth Moreland}, Christopher Sewell, Torsten Kuhlen, and E. Wes Bethel.
  Research Challenges for Visualization Software.
  \emph{IEEE Computer}, 5, 46(5):34--42, May 2013.
  doi:10.1109/MC.2013.179.
\item  % Moreland2013:TVCG
  \textbf{Kenneth Moreland}.
  A Survey of Visualization Pipelines.
  \emph{IEEE Transactions on Visualization and Computer Graphics}, 3, 19(3):367--378, March 2013.
  doi:10.1109/TVCG.2012.133.
\item  % Moreland2011:SC
  \textbf{Kenneth Moreland}, Wesley Kendall, Tom Peterka, and Jian Huang.
  An Image Compositing Solution at Scale.
  In \emph{Proceedings of 2011 International Conference for High Performance Computing, Networking, Storage and Analysis (SC '11)}, November 2011.
  doi:10.1145/2063384.2063417.
\item  % Biddiscombe2007
  John Biddiscombe, Berk Geveci, Ken Martin, \textbf{Kenneth Moreland}, and David Thompson.
  Time Dependent Processing in a Parallel Pipeline Architecture.
  \emph{IEEE Transactions on Visualization and Computer Graphics}, 6, 13(6):1376--1383, November/December 2007.
  doi:10.1109/TVCG.2007.70600.
\item  % Wylie2001
  Brian Wylie, Constantine Pavlakos, Vasily Lewis, and \textbf{Kenneth Moreland}.
  Scalable Rendering on {PC} Clusters.
  \emph{IEEE Computer Graphics and Applications}, 4, 21(4):62--70, July/August 2001.
\end{enumerate}

\subsection*{Symposium, Workshop, and Invited Papers}
\begin{enumerate}[label={[\arabic*]}, left=0pt]
\item  % Hari2024
  Gautam Hari, Nrushad Joshi, Zhe Wang, Qian Gong, Dave Pugmire, \textbf{Kenneth Moreland}, Chris R. Johnson, Scott Klasky, Norbert Podhorszki, and Tushar M. Athawale.
  {FunM$^2$C}: A Filter for Uncertainty Visualization of Multivariate Data on Multi-Core Devices.
  In \emph{Proceedings IEEE Workshop on Uncertainty Visualization}, pages 43--47, October 2024.
  doi:10.1109/UncertaintyVisualization63963.2024.00010.
\item  % Sisneros2024
  Robert Sisneros, Tushar Athawale, David Pugmire, and \textbf{Kenneth Moreland}.
  An Entropy-Based Test and Development Framework for Uncertainty Modeling in Level-Set Visualizations.
  In \emph{Proceedings IEEE Workshop on Uncertainty Visualization}, pages 78--83, October 2024.
  doi:10.1109/UncertaintyVisualization63963.2024.00015.
\item  % Hammer2024
  James Hammer, Tanner Hobson, David Pugmire, Scott Klasky, \textbf{Kenneth Moreland}, and Jian Huang.
  A Personalized {AI} Assistant For Intuition-Driven Visual Explorations.
  In \emph{IEEE 20th International Conference on e-Science (e-Science)}, September 2024.
  doi:10.1109/e-Science62913.2024.10678681.
\item  % Pugmire2024b
  David Pugmire, \textbf{Kenneth Moreland}, Tushar M. Athawale, James Hammer, and Jian Huang.
  Top Research Challenges and Opportunities for Near Real-Time Extreme-Scale Visualization of Scientific Data.
  In \emph{Proceedings IEEE 20th International Conference on e-Science}, September 2024.
  doi:10.1109/e-Science62913.2024.10678727.
\item  % Pugmire2024a
  David Pugmire, Jong Y. Choi, Scott Klasky, \textbf{Kenneth Moreland}, Eric Suchyta, Tushar M. Athawale, Zhe Wang, Choong-Seock Chang, Seung-Hoe Ku, and Robert Hager.
  Performance Improvements of {Poincar\'{e}} Analysis for Exascale Fusion Simulations.
  In \emph{VisGap - The Gap between Visualization Research and Visualization Software}, May 2024.
  doi:10.2312/visgap.20241120.
\item  % Tsalikis2024
  Spiros Tsalikis, Will Schroeder, Daniel Szafir, and \textbf{Kenneth Moreland}.
  An Accelerated Clip Algorithm for Unstructured Meshes: A Batch-Driven Approach.
  In \emph{Eurographics Symposium on Parallel Graphics and Visualization (EGPGV)}, May 2024.
  doi:10.2312/pgv.20241130.
\item  % Wang2023
  Zhe Wang, Tushar M. Athawale, \textbf{Kenneth Moreland}, Jieyang Chen, Chris R. Johnson, and David Pugmire.
  {FunMC$^2$}: A Filter for Uncertainty Visualization of Marching Cubes on Multi-Core Devices.
  In \emph{Eurographics Symposium on Parallel Graphics and Visualization (EGPGV)}, May 2023.
  doi:10.2312/pgv.20231081.
\item  % Moreland2022:InSitu
  \textbf{Kenneth Moreland}, Andrew C. Bauer, Berk Geveci, Patrick O'Leary, and Brad Whitlock.
  Leveraging Production Visualization Tools In Situ.
  In \emph{In Situ Visualization for Computational Science}, Springer, pages 205--231, 2022.
  doi:10.1007/978-3-030-81627-8\_10.
  ISBN:978-3-030-81626-1.
\item  % Ayachit2021
  Utkarsh Ayachit, Andrew C. Bauer, Ben Boeckel, Berk Geveci, \textbf{Kenneth Moreland}, Patrick O'Leary, and Tom Osika.
  Catalyst Revised: Rethinking the ParaView in Situ Analysis and Visualization {API}.
  In \emph{High Performance Computing}, pages 484--494, June 2021.
  doi:10.1007/978-3-030-90539-2\_33.
\item  % Sane2021
  Sudhanshu Sane, Abhishek Yenpure, Roxana Bujack, Matthew Larsen, \textbf{Kenneth Moreland}, Christoph Garth, Chris R. Johnson, and Hank Childs.
  Scalable In Situ Computation of {Lagrangian} Representations via Local Flow Maps.
  In \emph{Eurographics Symposium on Parallel Graphics and Visualization (EGPGV)}, June 2021.
Winner best paper.  doi:10.2312/pgv.20211040.
\item  % Lipinksi2021
  Riley Lipinksi, \textbf{Kenneth Moreland}, Michael E. Papka, and Thomas Marrinan.
  {GPU}-based Image Compression for Efficient Compositing in Distributed Rendering Applications.
  In \emph{2021 IEEE 11th Symposium on Large Data Analysis and Visualization (LDAV)}, pages 43--52, 2021.
  doi:10.1109/LDAV53230.2021.00012.
\item  % Yenpure2019
  Abhishek Yenpure, Hank Childs, and \textbf{Kenneth Moreland}.
  Efficient Point Merging Using Data Parallel Techniques.
  In \emph{Eurographics Symposium on Parallel Graphics and VIsualization (EGPGV)}, June 2019.
  doi:10.2312/pgv.20191112.
\item  % Moreland2018
  \textbf{Kenneth Moreland}.
  Comparing Binary-Swap Algorithms for Odd Factors of Processes.
  In \emph{Proceedings of the 8th IEEE Symposium on Large Data Analysis and Visualization (LDAV)}, October 2018.
  doi:10.1109/LDAV.2018.8739210.
\item  % Lessley2017:Duplicate
  Brenton Lessley, \textbf{Kenneth Moreland}, Matthew Larsen, and Hank Childs.
  Techniques for Data-Parallel Searching for Duplicate Elements.
  In \emph{IEEE Symposium on Large Data Analysis and Visualization (LDAV)}, October 2017.
  doi:10.1109/LDAV.2017.8231845.
\item  % Larsen2016
  Matthew Larsen, \textbf{Kenneth Moreland}, Chris Johnson, and Hank Childs.
  Optimizing Multi-Image Sort-Last Parallel Rendering.
  In \emph{Proceedings of the IEEE Symposium on Large Data Analysis and Visualization (LDAV)}, October 2016.
  doi:10.1109/LDAV.2016.7874308.
\item  % Moreland2016:HVEI
  \textbf{Kenneth Moreland}.
  Why We Use Bad Color Maps and What You Can Do About It.
  In \emph{Proceedings of Human Vision and Electronic Imaging (HVEI)}, February 2016.
  doi:10.2352/ISSN.2470-1173.2016.16.HVEI-133.
\item  % Ayachit2015
  Utkarsh Ayachit, Andrew Bauer, Berk Geveci, Patrick O'Leary, \textbf{Kenneth Moreland}, Nathan Fabian, and Jeffrey Mauldin.
  ParaView Catalyst: Enabling In Situ Data Analysis and Visualization.
  In \emph{Proceedings of the First Workshop on In Situ Infrastructures for Enabling Extreme-Scale Analysis and Visualization (ISAV 2015)}, pages 25--29, November 2015.
  doi:10.1145/2828612.2828624.
\item  % Moreland2015:SFI
  \textbf{Kenneth Moreland}, Matthew Larsen, and Hank Childs.
  Visualization for Exascale: Portable Performance is Critical.
  \emph{Supercomputing Frontiers and Innovations}, 3, 2(3), November 2015.
  doi:10.14529/jsfi150306.
\item  % Miller2014
  Robert Miller, \textbf{Kenneth Moreland}, and Kwan-Liu Ma.
  Finely-Threaded History-Based Topology Computation.
  In \emph{Eurographics Symposium on Parallel Graphics and Visualization}, June 2014.
  doi:10.2312/pgv.20141083.
\item  % Moreland2013:UltraVis
  \textbf{Kenneth Moreland}, Berk Geveci, Kwan-Liu Ma, and Robert Maynard.
  A Classification of Scientific Visualization Algorithms for Massive Threading.
  In \emph{Proceedings of Ultrascale Visualization Workshop}, November 2013.
  doi:10.1145/2535571.2535591.
\item  % Ayachit2013
  Utkarsh Ayachit, Berk Geveci, \textbf{Kenneth Moreland}, John Patchett, and Jim Ahrens.
  The ParaView Visualization Application  \emph{High Performance Visualization: Enabling Extreme Scale Insight}, CRC Press, pages 383--400, 2013.
  ISBN:978-1-4398-7572-8.
\item  % Moreland2013
  \textbf{Kenneth Moreland}.
  IceT  \emph{High Performance Visualization: Enabling Extreme Scale Insight}, CRC Press, pages 373--382, 2013.
  ISBN:978-1-4398-7572-8.
\item  % Moreland2012:PDAC
  \textbf{Kenneth Moreland}, Brad King, Robert Maynard, and Kwan-Liu Ma.
  Flexible Analysis Software for Emerging Architectures.
  In \emph{2012 SC Companion (Petascale Data Analytics: Challenges and Opportunities)}, pages 821--826, November 2012.
  doi:10.1109/SC.Companion.2012.115.
\item  % Moreland2012:LDAV
  \textbf{Kenneth Moreland}.
  Redirecting Research in Large-Format Displays for Visualization.
  In \emph{Proceedings of the IEEE Symposium on Large-Scale Data Analysis and Visualization}, pages 91--95, October 2012.
  doi:10.1109/LDAV.2012.6378981.
\item  % Moreland2011:PDAC
  \textbf{Kenneth Moreland}, Ron Oldfield, Pat Marion, Sebastien Jourdain, Norbert Podhorszki, Venkatram Vishwanath, Nathan Fabian, Ciprian Docan, Manish Parashar, Mark Hereld, Michael E. Papka, and Scott Klasky.
  Examples of {\it In Transit} Visualization.
  In \emph{Petascale Data Analytics: Challenges and Opportunities (PDAC-11)}, November 2011.
\item  % Fabian2011
  Nathan Fabian, \textbf{Kenneth Moreland}, David Thompson, Andrew C. Bauer, Pat Marion, Berk Geveci, Michel Rasquin, and Kenneth E. Jansen.
  The {ParaView} Coprocessing Library: A Scalable, General Purpose In Situ Visualization Library.
  In \emph{Proceedings of the IEEE Symposium on Large-Scale Data Analysis and Visualization}, pages 89--96, October 2011.
  doi:10.1109/LDAV.2011.6092322.
\item  % Moreland2011:LDAV
  \textbf{Kenneth Moreland}, Utkarsh Ayachit, Berk Geveci, and Kwan-Liu Ma.
  Dax Toolkit: A Proposed Framework for Data Analysis and Visualization at Extreme Scale.
  In \emph{Proceedings of the IEEE Symposium on Large-Scale Data Analysis and Visualization}, pages 97--104, October 2011.
  doi:10.1109/LDAV.2011.6092323.
\item  % Klasky2011
  Scott Klasky, \emph{et al}.
  In Situ Data Processing for Extreme Scale Computing.
  In \emph{Proceedings of SciDAC 2011}, July 2011.
\item  % Moreland2009
  \textbf{Kenneth Moreland}.
  Diverging Color Maps for Scientific Visualization.
  In \emph{Advances in Visual Computing (Proceedings of the 5th International Symposium on Visual Computing)}, 5876:92--103, December 2009.
  doi:10.1007/978-3-642-10520-3\_9.
\item  % Ma2009:SciDACReview
  Kwan-Liu Ma, Chaoli Wang, Hongfeng Yu, \textbf{Kenneth Moreland}, Jian Huang, and Rob Ross.
  Next-Generation Visualization Technologies: Enabling Discoveries at Extreme Scale.
  \emph{SciDAC Review}, 12, pages 12--21, Spring 2009.
\item  % Ross2008
  R B Ross, T Peterka, H-W Shen, Y Hong, K-L Ma, H Yu, and \textbf{K Moreland}.
  Visualization and parallel {I/O} at extreme scale.
  \emph{Journal of Physics: Conference Series}, 012099, 125(012099), July 2008.
  doi:10.1088/1742-6596/125/1/012099.
\item  % Ma2007
  Kwan-Liu Ma, Robert Ross, Jian Huang, Greg Humphreys, Nelson Max, \textbf{Kenneth Moreland}, John D. Owens, and Han-Wei Shen.
  Ultra-Scale Visualization: Research and Education.
  \emph{Journal of Physics: Conference Series}, 012088, 78(012088), June 2007.
  doi:10.1088/1742-6596/78/1/012088.
\item  % Moreland2007
  \textbf{Kenneth Moreland}, Lisa Avila, and Lee Ann Fisk.
  Parallel Unstructured Volume Rendering in ParaView.
  In \emph{Visualization and Data Analysis 2007, Proceedings of SPIE-IS\&T Electronic Imaging}, pages 64950F-1--12, January 2007.
\item  % Cedilnik2006
  Andy Cedilnik, Berk Geveci, \textbf{Kenneth Moreland}, James Ahrens, and Jean Farve.
  Remote Large Data Visualization in the {ParaView} Framework.
  In \emph{Eurographics Parallel Graphics and Visualization}, pages 163--170, May 2006.
\item  % Moreland2004:VolVis
  \textbf{Kenneth Moreland} and Edward Angel.
  A Fast High Accuracy Volume Renderer for Unstructured Data.
  In \emph{IEEE Symposium on Volume Visualization and Graphics}, pages 9--16, October 2004.
  doi:10.1109/SVVG.2004.2.
\item  % Moreland2003
  \textbf{Kenneth Moreland} and David Thompson.
  From Cluster to Wall with {VTK}.
  In \emph{Proceedings of IEEE Symposium on Parallel and Large-Data Visualization and Graphics}, pages 25--31, October 2003.
\item  % Moreland2003:FFT
  \textbf{Kenneth Moreland} and Edward Angel.
  {The FFT on a GPU}.
  In \emph{SIGGRAPH/Eurographics Workshop on Graphics Hardware 2003 Proceedings}, pages 112--119, July 2003.
\item  % Wylie2002
  Brian Wylie, \textbf{Kenneth Moreland}, Lee Ann Fisk, and Patricia Crossno.
  Tetrahedral Projection using Vertex Shaders.
  In \emph{Proceedings of IEEE Volume Visualization and Graphics Symposium}, pages 7--12, October 2002.
\item  % Moreland2001
  \textbf{Kenneth Moreland}, Brian Wylie, and Constantine Pavlakos.
  Sort-Last Parallel Rendering for Viewing Extremely Large Data Sets on Tile Displays.
  In \emph{Proceedings of the IEEE 2001 Symposium on Parallel and Large-Data Visualization and Graphics}, pages 85--92, October 2001.
  doi:10.1109/PVGS.2001.964408.
\end{enumerate}

\subsection*{Technical Reports and other Non Peer Reviewed}
% DO NOT EDIT THIS FILE!
% This content is automatically generated from nopeer.bib.

\begin{enumerate}[label={[\arabic*]}, left=0pt]
\item  % ViskoresUsersGuide10
  \textbf{Kenneth Moreland}.
  The Viskores User's Guide, Release 1.0.
Technical Report ORNL/TM-2025/4019, Oak Ridge National Laboratory, April 2025.
\item  % Samsel2024:KWSource
  Francesca Samsel, \textbf{Ken Moreland}, W. Alan Scott, Spiros Tsalikis, and Cory Quammen.
  New Default Colormap and Background in the next version of {ParaView}.
\emph{Kitware Source}, October 2024.
\item  % VTKmUsersGuide22
  \textbf{Kenneth Moreland}.
  The {VTK-m} User's Guide, {VTK-m} version 2.2.
Technical Report ORNL/TM-2024/3443, Oak Ridge National Laboratory, July 2024.
\item  % VTKmUsersGuide21
  \textbf{Kenneth Moreland}.
  The {VTK-m} User's Guide, {VTK-m} version 2.1.
Technical Report ORNL/TM-2023/3182, Oak Ridge National Laboratory, November 2023.
\item  % VTKmUsersGuide20
  \textbf{Kenneth Moreland}.
  The {VTK-m} User's Guide, {VTK-m} version 2.0.
Technical Report ORNL/TM-2023/2863, Oak Ridge National Laboratory, February 2023.
\item  % Philip2023
  Sujin Philip, \textbf{Kenneth Moreland}, and Robert Maynard.
  {VTK-m} Accelerated Filters in {VTK} and {ParaView}.
\emph{Kitware Source}, 2023.
\item  % VTKmUsersGuide19
  \textbf{Kenneth Moreland}.
  The {VTK-m} User's Guide, {VTK-m} version 1.9.
Technical Report ORNL/TM-2022/2744, Oak Ridge National Laboratory, October 2022.
\item  % Moreland2022:NovelHardware
  \textbf{Kenneth Moreland}, David Pugmire, Berk Geveci, Li-Ta Lo, Hank Childs, Mark Bolstad, Ruchi Shah, and Panruo Wu.
  The Importance of Scientific Visualization on Novel Hardware.
Technical Report ORNL/LTR-2022/415, Oak Ridge National Laboratory, September 2022.
\item  % VTKmUsersGuide18
  \textbf{Kenneth Moreland}.
  The {VTK-m} User's Guide, {VTK-m} version 1.8.
Technical Report ORNL/TM-2022/2516, Oak Ridge National Laboratory, August 2022.
\item  % Moreland2022:DataReduceTR
  \textbf{Kenneth Moreland}, David Pugmire, and Jieyang Chen.
  The Exploitation of Data Reduction for Visualization.
Technical Report ORNL/LTR-2022/412, Oak Ridge National Laboratory, August 2022.
\item  % VTKmUsersGuide17
  \textbf{Kenneth Moreland}.
  The {VTK-m} User's Guide, {VTK-m} version 1.7.
Technical Report ORNL/TM-2021/2346, Oak Ridge National Laboratory, December 2021.
\item  % VTKmUsersGuide16
  \textbf{Kenneth Moreland}.
  The {VTK-m} User's Guide, {VTK-m} version 1.6.
Technical Report ORNL/TM-2021/2075, Oak Ridge National Laboratory, July 2021.
\item  % VTKmUsersGuide15
  \textbf{Kenneth Moreland}.
  The {VTK-m} User's Guide, {VTK-m} version 1.5.
Technical Report SAND 2019-12638 B, Sandia National Laboratories, October 2019.
\item  % Moreland2019:XVis
  \textbf{Kenneth Moreland}, David Pugmire, David Rogers, Hank Childs, Kwan-Liu Ma, and Berk Geveci.
  {XVis}: Visualization for the Extreme-Scale Scientific Computation Ecosystem, Final Report.
Technical Report SAND 2019-9297, Sandia National Laboratories, August 2019.
  doi:10.2172/1762947.
\item  % VTKmUsersGuide14
  \textbf{Kenneth Moreland}.
  The {VTK-m} User's Guide, {VTK-m} version 1.4.
Technical Report SAND 2019-8008 B, Sandia National Laboratories, July 2019.
\item  % Bennett2019
  Janine C. Bennett, Hank Childs, Christoph Garth, Bernd Hentschel, \emph{et al}.
  In Situ Visualization for Computational Science (Dagstuhl Seminar 18271).
  \emph{Dagstuhl Reports}, 7, 8(7):1--43, 2019.
  doi:10.4230/DagRep.8.7.1.
\item  % VTKmUsersGuide13
  \textbf{Kenneth Moreland}.
  The {VTK-m} User's Guide, {VTK-m} version 1.3.
Technical Report SAND 2018-13465 B, Sandia National Laboratories, November 2018.
\item  % Bourgeois2018
  Daniel Bourgeois, Michael Wolf, and \textbf{Kenneth Moreland}.
  Isosurface Visualization Miniapplication.
Technical Report SAND2018-2780O, Sandia National Laboratories, 2018.
\item  % Moreland2018:HPCManage
  \textbf{Kenneth Moreland} and Chuck Atkins.
  A Need for Better Management of Heterogenous HPC Resources.
Technical Report Sandia National Laboratories, 2018.
\item  % ExascaleASCRReport
  Jeffrey Vetter, Ann Almgren, Phil DeMar, Katherine Riley, Katie Antypas, Deborah Bard, Richard Coffey, Eli Dart, Sudip Dosanjh, Richard Gerber, James Hack, Inder Monga, Michael E. Papka, Lauren Rotman, Tjerk Straatsma, Jack Wells, David E. Bernholdt, Wes Bethel, George Bosilca, Frank Cappello, Todd Gamblin, Salman Habib, Judy Hill, Jeffrey K. Hollingsworth, Lois Curfman McInnes, Kathryn Mohror, Shirley Moore, \textbf{Ken Moreland}, Rob Roser, Sameer Shende, Galen Shipman, and Samuel Williams.
  Advanced Scientific Computing Research Exascale Requirements Review.
Technical Report An Office of Science review sponsored by Advanced Scientific Computing Research, September 2016.
  doi:10.2172/1375638.
\item  % Workflows2015
  Ewa Dellman, Tom Peterka, \emph{et al}.
  The Future of Scientific Workflows.
\emph{Report of the DOE NGNS/CS Scientific Workflows Workshop}, April 2015.
\item  % ParaViewGuide
  Utkarsh Ayachit.
  \emph{The {ParaView} Guide: A Parallel Visualization Application}.
Kitware Inc., January 2015.
(contributions).  ISBN:978-1-930934-30-6.
\item  % Moreland2015
  \textbf{Kenneth Moreland}.
  The {ParaView} Tutorial, Version 4.4.
Technical Report SAND2015-7813 TR, 2015.
\item  % Schroots2014
  Hendrik Schroots and \textbf{Kenneth Moreland}.
  Implementing Parallel Algorithms Using the Dax Toolkit.
Technical Report SAND 2015-3829 O, Sandia National Laboratories, December 2014.
CR Summer Proceedings.\item  % Moreland2014:DaxFinal
  \textbf{Kenneth Moreland}.
  A Pervasive Parallel Framework for Visualization: Final Report for FWP 10-014707.
Technical Report SAND 2014-0047, Sandia National Laboratories, January 2014.
\item  % Rogers2013
  David Rogers, \textbf{Kenneth Moreland}, Ron Oldfield, and Nathan Fabian.
  Data Co-Processing for Extreme Scale Analysis Level {II} {ASC} Milestone (4745).
Technical Report SAND2013-1122, Sandia National Laboratories, 2013.
\item  % Moreland2012:TR
  \textbf{Kenneth Moreland}, Jeremy Meredith, and Berk Geveci.
  Enabling Production-Quality Scientific-Discovery Tools with Data and Execution Models.
Technical Report SAND 2012-10796P, Sandia National Laboratories, December 2012.
\item  % Moreland2012:Ultravis
  \textbf{Kenneth Moreland}.
  {Oh, \$\#*@! Exascale!} {The} Effect of Emerging Architectures on Scientific Discovery.
  In \emph{2012 SC Companion (Proceedings of the Ultrascale Visualization Workshop)}, pages 224--231, November 2012.
  doi:10.1109/SC.Companion.2012.38.
\item  % Sewell2012
  Christopher Sewell, Jeremy Meredith, \textbf{Kenneth Moreland}, Tom Peterka, Dave DeMarle, Li-Ta Lo, James Ahrens, Robert Maynard, and Berk Geveci.
  The {SDAV} Software Frameworks for Visualization and Analysis on Next-Generation Multi-Core and Many-Core Architectures.
  In \emph{2012 SC Companion (Proceedings of the Ultrascale Visualization Workshop)}, pages 206--214, November 2012.
  doi:10.1109/SC.Companion.2012.36.
\item  % Barrett2012
  Brian Barrett, Richard Barrett, James Brandt, Ron Brightwell, Matthew Curry, Nathan Fabian, Kurt Ferreira, Ann Gentile, Scott Hemmert, Suzanne Kelly, Ruth Klundt, James Laros III, Vitus Leung, Michael Levenhagen, Gerald Lofstead, \textbf{Ken Moreland}, Ron Oldfield, Kevin Pedretti, Arun Rodrigues, David Thompson, Tom Tucker, Lee Ward, John Van Dyke, Courtenay Vaughan, and Kyle Wheeler.
  Report of Experiments and Evidence for ASC L2 Milestone 4467 - Demonstration of a Legacy Application’s Path to Exascale.
Technical Report SAND 2012-1750, Sandia National Laboratories, March 2012.
\item  % IceT
  \textbf{Kenneth Moreland}.
  {IceT} Users' Guide and Reference, Version 2.1.
Technical Report SAND2011-5011, Sandia National Laboratories, August 2011.
\item  % ScientificDiscoveryExascale2011
  Sean Ahern, Arie Shoshani, Kwan-Liu Ma, \emph{et al}.
  Scientific Discovery at the Exascale.
\emph{Report from the DOE ASCR 2011 Workshop on Exascale Data Management, Analysis, and Visualization}, February 2011.
\item  % Moreland2010:IceT
  \textbf{Kenneth Moreland}.
  IceT Users’ Guide and Reference, Version 2.0.
Technical Report SAND2010-7451, Sandia National Laboratories, December 2010.
  doi:10.2172/1005031.
\item  % Moreland2010
  \textbf{Kenneth Moreland}, Nathan Fabian, Pat Marion, and Berk Geveci.
  Visualization on Supercomputing Platform Level {II} {ASC} Milestone {(3537-1B)} Results from {Sandia}.
Technical Report SAND 2010-6118, Sandia National Laboratories, September 2010.
\item  % Ice2009
  Lisa Ice, Nathan Fabian, \textbf{Kenneth D. Moreland}, Janine C. Bennett, David C. Thompson, David B. Karelitz, and W. Alan Scott.
  Scalable Analysis Tools for Sensitivity Analysis and {UQ} (3160) Results.
Technical Report 2009-6032, Sandia National Laboratories, September 2009.
\item  % Thompson2009
  David Thompson, Nathan D. Fabian, \textbf{Kenneth D. Moreland}, and Lisa G. Ice.
  Design Issues for Performing {\textit In Situ} Analysis of Simulation Data.
Technical Report SAND2009-2014, Sandia National Laboratories, 2009.
\item  % Moreland2008:UltraVis
  \textbf{Kenneth Moreland}, C. Charles Law, Lisa Ice, and David Karelitz.
  Analysis of Fragmentation in Shock Physics Simulation.
  In \emph{Proceedings of the 2008 Workshop on Ultrascale Visualization}, pages 40-46, November 2008.
  doi:10.1109/ULTRAVIS.2008.5154062.
\item  % Karelitz2008
  David B. Karelitz, Lisa Ice, Jason Wilke, Stephen W. Attaway, and \textbf{Kenneth Moreland}.
  Post-Processing {V\&V} Level {II} {ASC} Milestone (2843) Results.
Technical Report SAND 2008-6183, Sandia National Laboratories, September 2008.
\item  % Moreland2008:CUG
  \textbf{Kenneth Moreland}, David Rogers, John Greenfield, Berk Geveci, Patrick Marion, Alexander Neundorf, and Kent Eschenberg.
  Large Scale Visualization on the {Cray XT3} Using ParaView.
  In \emph{Cray User Group}, 2008.
\item  % Crossno2007
  Patricia Crossno, Brian Wylie, Andrew Wilson, John Greenfield, Eric Stanton, Timothy Shead, Lisa Ice, \textbf{Kenneth Moreland}, Jeffrey Baumes, and Berk Geveci.
  Intelligence Analysis Using Titan.
  In \emph{IEEE Symposium on Visual Analytics Science and Technology}, October 2007.
\item  % Moreland2007:TR
  \textbf{Kenneth Moreland} and Brian Wylie.
  Massive Graph Visualization: LDRD Final Report.
Technical Report SAND 2007-6307, Sandia National Laboratories, October 2007.
\item  % Karelitz2007
  David B. Karelitz, Elmer Chavez, V. Gregory Weirs, Timothy M. Shead, \textbf{Kenneth D. Moreland}, Thomas A. Brunner, and Timothy G. Trucano.
  Post-Processing {V\&V} Level {II} {ASC} Milestone (2360) Results.
Technical Report SAND 2007-6006, Sandia National Laboratories, September 2007.
\item  % Moreland2006:KWS
  \textbf{Kenneth Moreland}.
  Using Ghost Cells in Parallel Filters.
  \emph{Kitware Source}, pages 3--4, July 2006.
\end{enumerate}

\subsection*{Posters}
\begin{enumerate}[label={[\arabic*]}, left=0pt]
\item  % Duggan2025:poster
  John Duggan, Dave Pugmire, \textbf{Ken Moreland}, Greg Watson, Sergey Yakubov, Greg Cage, and Andrew Ayres.
  Visualizing Imaging Data with {NDIP} and {NOVA}.
  In \emph{Joint Nanoscience and Neutron Scattering User Meeting}, August 2025.
\item  % Yakubov2025:Poster
  Sergey Yakubov, Greg Watson, Greg Cage, John Duggan, Andrew Ayres, Dave Pugmire, \textbf{Ken Moreland}, Kristin Maroun, and Randall Petras.
  {NDIP} and {NOVA}: Workflows and Interfaces for Neutron Scattering.
  In \emph{Joint Nanoscience and Neutron Scattering User Meeting}, August 2025.
\item  % Maynard2013:poster
  Robert Maynard, \textbf{Kenneth Moreland}, Utkarsh Ayachit, Berk Geveci, and Kwan-Liu Ma.
  Optimizing Threshold for Extreme Scale Analysis.
  In \emph{Proceedings of SPIE Visualization and Data Analysis}, 2013.
\item  % Moreland2011:poster
  \textbf{Kenneth Moreland}, Utkarsh Ayachit, Berk Geveci, and Kwan-Liu Ma.
  Dax: Data Analysis at Extreme.
  In \emph{Proceedings of SciDAC}, July 2011.
\item  % Peterka2010:poster
  T Peterka, W Kendall, D Goodell, B Nouanesengsey, H-W Shen, J Huang, \textbf{K Moreland}, R Thakur, and R B Ross.
  Performance of communication patterns for extreme-scale analysis and visualization.
  In \emph{Proceedings of SciDAC}, 2010.
\item  % Moreland2008:poster
  \textbf{Kenneth Moreland}, Daniel Lepage, David Koller, and Greg Humphreys.
  Remote rendering for ultrascale data.
  \emph{Journal of Physics: Conference Series}, 012096, 125(012096), 2008.
\end{enumerate}


\subsection*{Presentations}
\begin{enumerate}[label={[\arabic*]}, left=0pt]
\item  % Moreland2025:HPSF
  \textbf{Kenneth Moreland}.
  Viskores Update.
  \emph{HPSF Conference}, May 5, 2025.
\item  % Wang2025
  Zhe Wang, \textbf{Kenneth Moreland}, Matthew Larsen, James Kress, Hank Childs, and David Pugmire.
  Parallelize Over Data Particle Advection: Participation, Ping Pong Particles, and Overhead.
  \emph{DOE Computer Graphics Forum}, May 1, 2025.
\item  % Moreland2025:DOECGF:Viskores
  \textbf{Kenneth Moreland}.
  Viskores Update.
  \emph{DOE Computer Graphics Forum}, April 29, 2025.
\item  % Moreland2025:ADIOS
  \textbf{Kenneth Moreland}, Qian Gong, and Norbert Podhorszki.
  Compression Support with ADIOS.
  \emph{Data-driven Plasma Science and Engineering Workshop}, February 27, 2025.
\item  % Moreland2024:BestPractice
  \textbf{Kenneth Moreland}.
  Development of VTK-m During ECP.
  \emph{HPC Best Practices Webinar}, September 25, 2024.
\item  % Moreland2024:PVTut
  \textbf{Kenneth Moreland}.
  ParaView on Frontier Training.
August 29, 2024.
\item  % Moreland2024:CASS
  \textbf{Kenneth Moreland}.
  Using VTK-m in ParaView and Catalyst.
  \emph{ParaView and Catalyst CASS Community BOF}, June 12, 2024.
\item  % Moreland2024:NASA
  \textbf{Kenneth Moreland}, Scott Klasky, Norbert Podhorszki, and Qian Gong.
  Data Management, Workflow, and Visualization Solutions at ORNL.
  \emph{Software for the NASA Science Mission Directorate Workshop}, May 8, 2024.
\item  % Moreland2024:DOECGF:VTKm
  \textbf{Kenneth Moreland}.
  VTK-m Update.
  \emph{DOE Computer Graphics Forum}, April 23, 2024.
\item  % Moreland2023:LDAV
  \textbf{Kenneth Moreland}.
  Enabling Visualization at the Exascale with VTK-m.
  \emph{IEEE Symposium on Large Data Analysis and Visualization (LDAV)}, October 23, 2023.
Keynote.\item  % Bolstad2023
  Mark Bolstad, \textbf{Kenneth Moreland}, David Pugmire, David Rogers, Li-Ta Lo, Berk Geveci, Hank Childs, and Silvio Rizzi.
  VTK-m: Visualization for the Exascale Era and Beyond.
  \emph{ACM SIGGRAPH 2023 Talks}, August 2023.
\item  % Moreland2023:Color
  \textbf{Kenneth Moreland}.
  Color: What It Is and How to Use It to Show Data.
  \emph{LANL Data Science Summer School Seminar}, Los Alamost National Laboratory, July 18, 2023.
\item  % Moreland2023:VTKmUpdate
  \textbf{Kenneth Moreland}.
  VTK-m Update.
  \emph{DOE Computer Graphics Forum}, April 26, 2023.
\item  % Moreland2023:VisTools
  \textbf{Kenneth Moreland} and David Pugmire.
  DOE Visualization Tools and Capabilities.
  \emph{PSI2 SciDAC All-Hands}, February 22, 2023.
\item  % Moreland2023:Portability
  \textbf{Kenneth Moreland}.
  Performance Portability Pre-ECP and Post-ECP -- VTK-m.
  \emph{ECP Annual Meeting}, January 19, 2023.
\item  % Athawale2022:VTKmTutorial
  Tushar M. Athawale, \textbf{Kenneth Moreland}, David Pugmire, Silvio Rizzi, and Mark Bolstad.
  VTK-m -- A ToolKit for Scientific Visualization on Many-Core Processors.
  \emph{IEEE VIS Tutorials}, October 17, 2022.
\item  % Moreland2022:VisSC
  \textbf{Kenneth Moreland}.
  Scientific Visualization on Supercomputers.
  \emph{D\&AI Section All Hands Meeting}, Oak Ridge National Laboratory, September 21, 2022.
\item  % Moreland2022:OLCFTut
  \textbf{Kenneth Moreland}.
  ParaView OLCF Tutorial.
September 15, 2022.
\item  % Moreland2022:VTKmUpdate
  \textbf{Kenneth Moreland}.
  VTK-m Update.
  \emph{DOE Computer Graphics Forum}, August 31, 2022.
\item  % Moreland2022:ECP
  \textbf{Kenneth Moreland}.
  ECP Data Management, Data Analytics, and Visualization Overview: VTK-m.
  \emph{ECP Annual Meeting}, May 2, 2022.
\item  % Moreland2022:SCI
  \textbf{Kenneth Moreland}.
  VTK-m: Accelerating Visualization on Massively Threaded Processors.
  \emph{SCI Institute Visualization Seminar}, University of Utah, February 2, 2022.
\item  % Moreland2021:VTKmOverview
  \textbf{Kenneth Moreland}.
  VTK-m: What it is, Why we need it, and How to use it.
November 2021.
\item  % Moreland2021:Color
  \textbf{Kenneth Moreland}.
  Color: What it is and How to Use It to Show Data.
  \emph{VISTA Webinar}, Oak Ridge National Laboratory, September 9, 2021.
\item  % Moreland2021:RAPIDS
  \textbf{Kenneth Moreland}.
  Enabling Visualization on Exascale Accelerators with VTK-m.
  \emph{RAPIDS2-DU Monthly Meeting}, July 14, 2021.
\item  % Moreland2021:DOECGFVTKm
  \textbf{Kenneth Moreland}.
  VTK-m Update.
  \emph{DOE Computer Graphics Forum}, April 28, 2021.
\item  % Moreland2021:ECP
  \textbf{Kenneth Moreland}.
  ECP Data Management, Data Analytics, and Visualization Overview: VTK-m.
  \emph{ECP Annual Meeting}, March 30, 2021.
\item  % Moreland2021:ORNL
  \textbf{Kenneth Moreland}.
  A Winding Road to Exascale Visualization.
  \emph{Oak Ridge National Laboratories}, February 2, 2021.
\item  % Moreland2020:DOECGFPV
  \textbf{Kenneth Moreland}.
  What's New in ParaView.
  \emph{DOE Computer Graphics Forum}, April 28, 2020.
\item  % Moreland2020:DOECGFVTKm
  \textbf{Kenneth Moreland}.
  VTK-m Update.
  \emph{DOE Computer Graphics Forum}, April 28, 2020.
\item  % Moreland2020:ECP
  \textbf{Kenneth Moreland}.
  In Situ Visualization and Analysis with Ascent: Using VTK-m.
  \emph{ECP Annual Meeting Tutorials}, February 4, 2020.
\item  % Childs2019
  Hank Childs, \textbf{Kenneth Moreland}, David Pugmire, and Robert Maynard.
  VTK-m -- A ToolKit for Scientific Visualization on Many-Core Processors.
  \emph{IEEE VIS Tutorials}, October 2019.
\item  % Moreland2019:VisSandia
  \textbf{Kenneth Moreland}.
  Vis Capabilities at Sandia.
  \emph{Informal Review for 1540 Production Codes}, Sandia National Laboratories, June 10, 2019.
\item  % Moreland2019:DOECGFPV
  \textbf{Kenneth Moreland}.
  What's New in ParaView.
  \emph{DOE Computer Graphics Forum}, April 23, 2019.
\item  % Moreland2019:DOECGFVTKm
  \textbf{Kenneth Moreland}.
  VTK-m Update.
  \emph{DOE Computer Graphics Forum}, April 23, 2019.
\item  % Moreland2019:ECP
  \textbf{Kenneth Moreland}.
  In Situ Visualization and Analysis with Ascent: Using VTK-m.
  \emph{ECP Annual Meeting Tutorials}, January 17, 2019.
\item  % Moreland2018:Plots
  \textbf{Kenneth Moreland}.
  Building Better Plots.
  \emph{CCR Summer Seminar Talks}, Sandia National Laboratories, July 17, 2018.
\item  % Moreland2018:Stuttgart
  \textbf{Kenneth Moreland}.
  A Brief History of Interactive Visualization.
  \emph{University of Stuttgart Colloquium}, June 29, 2018.
\item  % Moreland2018:ExPrep
  \textbf{Kenneth Moreland}.
  Preparations for Exascale Visualization at DOE.
  \emph{ISC High Performance}, June 25, 2018.
\item  % Moreland2018:DOECGFPV
  \textbf{Kenneth Moreland}.
  What's New in ParaView.
  \emph{DOE Computer Graphics Forum}, April 24, 2018.
\item  % Moreland2018:DOECGFVTKm
  \textbf{Kenneth Moreland}.
  VTK-m Update.
  \emph{DOE Computer Graphics Forum}, April 24, 2018.
\item  % Moreland2018:InSitu
  \textbf{Kenneth Moreland}.
  The Difficulties and Indispensability of In Situ Visualization.
  \emph{Applied Computer Science (ACS) Technical Exchange Meeting}, February 13, 2018.
\item  % Moreland2017:SCBooth
  \textbf{Kenneth Moreland}.
  VTK-m: Visualization on Modern Processors.
  \emph{Kitware Booth, SC17}, November 15, 2017.
\item  % Moreland2017:SCTut
  \textbf{Kenneth Moreland}, W. Alan Scott, David E. DeMarle, Joe Insley, John Patchett, and Jon Woodring.
  Large Scale Visualization with ParaView.
  \emph{Supercomputing Tutorials}, November 14, 2017.
\item  % Moreland2017:InSituSNL
  \textbf{Kenneth Moreland}.
  Why You \emph{Don't} Want to do In Situ Visualization, and Why You Have To.
  \emph{Computational Science Seminar Series}, Sandia National Laboratories, October 31, 2017.
\item  % Moreland2017:PNNL
  \textbf{Kenneth Moreland}.
  High Performance Visualization in the Many-Core Era.
  \emph{Computing@PNNL Seminar}, Pacific Northwest National Laboratory, August 11, 2017.
\item  % Moreland2017:InSituISC
  \textbf{Kenneth Moreland}.
  Why You \emph{Don't} Want to do In Situ Visualization, and Why You Have To.
  \emph{ISC Workshop on In Situ Visualization}, June 22, 2017.
\item  % Moreland2017:ISCScale
  \textbf{Kenneth Moreland}.
  The Many Faces and Solutions of In Situ Visualization.
  \emph{ISC Workshop on Visualization at Scale}, June 22, 2017.
\item  % Moreland2017:Plots
  \textbf{Kenneth Moreland}.
  Building Better Plots.
  \emph{Sandia Post-Doc Workshop}, Sandia National Laboratories, February 20, 2017.
\item  % Moreland2016:SCTut
  \textbf{Kenneth Moreland}, W. Alan Scott, David E. DeMarle, Joe Insley, Jonathan Woodring, and John Patchett.
  Large Scale Visualization with ParaView.
  \emph{Supercomputing Tutorials}, November 14, 2016.
\item  % Moreland2015:UltraVis
  \textbf{Kenneth Moreland}.
  VTK-m: Building a Visualization Toolkit for Massively Threaded Architectures.
  \emph{Ultrascale Visualization Workshop}, November 17, 2015.
\item  % Moreland2015:SCTut
  \textbf{Kenneth Moreland}, W. Alan Scott, David DeMarle, Li-Ta Lo, Joseph Insley, and Rich Cook.
  Large Scale Visualization with ParaView.
  \emph{Supercomputing Tutorials}, November 15, 2015.
\item  % Moreland2015:LANL
  \textbf{Kenneth Moreland}.
  High Performance Visualization in the Many-Core Era.
  \emph{LANL Information Science and Technology Seminar Series}, July 1, 2015.
\item  % Moreland2014:Interactive
  \textbf{Kenneth Moreland}.
  Maintaining Interactivity in ParaView.
  \emph{Kitware SC Booth}, November 19, 2014.
\item  % Moreland2014:HPC
  \textbf{Kenneth Moreland}.
  HPC Processing in ParaView.
  \emph{Kitware SC Booth}, November 18, 2014.
\item  % Moreland2014:SCTut
  \textbf{Kenneth Moreland}, W. Alan Scott, David DeMarle, Li-Ta Lo, Joseph Insley, and Robert Maynard.
  Large Scale Visualization with ParaView.
  \emph{Supercomputing Tutorials}, November 16, 2014.
\item  % Moreland2014:NERSC
  \textbf{Kenneth Moreland}.
  Using ParaView.
  \emph{NERSC Joint Facilities User Forum on Data-Intensive Computing}, June 18, 2014.
\item  % Moreland2014:GTC
  \textbf{Kenneth Moreland} and Robert Maynard.
  Dax: A Massively Threaded Visualization and Analysis Toolkit for Extreme Scale.
  \emph{GPU Technology Conference}, March 26, 2014.
\item  % Moreland2014:SIAM
  \textbf{Kenneth Moreland}, Ron Oldfield, Nathan Fabian, Berk Geveci, Andrew Bauer, and David Lonie.
  Approaching Production In Situ Visualization for Extreme Scale Analysis.
  \emph{16th SIAM Conference on Parallel Processing for Scientific Computing}, February 21, 2014.
\item  % Moreland2014:UO
  \textbf{Kenneth Moreland}.
  15 Years of Large-Scale Scientific Visualization.
  \emph{University of Oregon Colloquium}, January 23, 2014.
\item  % Moreland2014:Color
  \textbf{Kenneth Moreland}.
  Using Color in Your Diagrams and Presentations.
  \emph{Sandia Center 2700 Lunch and Learn Program}, 2014.
\item  % Moreland2013:SCPV
  \textbf{Kenneth Moreland}, W. Alan Scott, David DeMarle, and Li-Ta Lo.
  Large Scale Visualization with ParaView.
  \emph{Supercomputing Tutorials}, November 17, 2013.
\item  % Moreland2012:SCPV
  \textbf{Kenneth Moreland}, W. Alan Scott, Nathan Fabian, Utkarsh Ayachit, and Robert Maynard.
  Large Scale Visualization with ParaView.
  \emph{Supercomputing Tutorial}, November 2012.
\item  % Moreland2012:SIAM
  \textbf{Kenneth Moreland}, Nathan Fabian, Berk Geveci, Utkarsh Ayachit, and James Ahrens.
  Next-Generation Capabilities for Large-Scale Scientific Visualization.
  \emph{15th SIAM Conference on Parallel Processing for Scientific Computing}, February 15, 2012.
\item  % Moreland2011:UltraVis
  \textbf{Kenneth Moreland}, Nathan Fabian, Scott Klasky, and Berk Geveci.
  Flexible In Situ with ParaView.
  \emph{Workshop on Ultrascale Visualization}, November 13, 2011.
\item  % Moreland2011:SCPV
  \textbf{Kenneth Moreland}, W. Alan Scott, Nathan Fabian, Jeffrey Mauldin, Andrew C. Bauer, Robert Maynard, and Scott Klasky.
  Large Scale Visualization with ParaView.
  \emph{Supercomputing Tutorials}, November 2011.
\item  % Moreland2011:ICCE
  \textbf{Kenneth Moreland}.
  Large-Scale Interactive Visualization with ParaView.
  \emph{2nd International Conference on Computational Engineering}, October 6, 2011.
\item  % Moreland2011:SciDAC
  \textbf{Kenneth Moreland}.
  Large Scale Visualization with ParaView.
  \emph{SciDAC Tutorials}, July 2011.
\item  % Moreland2010:SCInSitu
  \textbf{Kenneth Moreland}, Andrew Bauer, Pat Marion, and Nathan Fabian.
  In-Situ Visualization with the ParaView Coprocessing Library.
  \emph{Supercomputing Tutorials}, November 2010.
\item  % Moreland2010:SciDAC
  \textbf{Kenneth Moreland} and Andrew Bauer.
  Large Scale Visualization with ParaView.
  \emph{SciDAC Tutorials}, July 2010.
\item  % Moreland2009:SCPV
  \textbf{Kenneth Moreland}, John Greenfield, W. Alan Scott, Utkarsh Ayachit, and Berk Geveci.
  Large Scale Visualization with ParaView.
  \emph{Supercomputing Tutorials}, November 2009.
\item  % Moreland2009:Cluster
  \textbf{Kenenth Moreland} and David DeMarle.
  Parallel Distributed-Memory Visualization with ParaView.
  \emph{IEEE Cluster Tutorials}, August 2009.
\item  % Moreland2008:SCPV
  \textbf{Kenneth Moreland}, John Greenfield, W. Alan Scott, Utkarsh Ayachit, Berk Geveci, and David DeMarle.
  Large Scale Visualization with ParaView.
  \emph{Supercomputing Tutorials}, November 2008.
\item  % Moreland2008:VisTut
  \textbf{Kenneth Moreland}, Utkarsh Ayachit, Timothy Shead, John Biddiscombe, and David Thompson.
  Advanced ParaView Visualization.
  \emph{IEEE Visualization Tutorials}, October 2008.
\item  % Moreland2007:SCPV
  \textbf{Kenneth Moreland} and John Greenfield.
  Large Scale Visualization with ParaView 3.
  \emph{Supercomputing Tutorials}, November 2007.
\item  % Moreland2005:SCPV
  \textbf{Kenneth Moreland} and Berk Geveci.
  Parallel Visualization with ParaView.
  \emph{Supercomputing Tutorials}, November 2005.
\item  % Moreland2004:LCIW
  \textbf{Kenneth Moreland}.
  Large Scale Visualization with Cluster Computing.
  \emph{Linux Cluster Institute Workshop}, October 1, 2004.
\item  % Moreland2002:UNM
  \textbf{Kenneth Moreland}.
  Big Data, Big Displays, and Cluster-Driven Interactive Visualization.
  \emph{University of New Mexico Colloquium}, November 2002.
\item  % Moreland2002:CBVC
  \textbf{Kenneth Moreland}.
  Big Data, Big Displays, and Cluster-Driven Interactive Visualization.
  \emph{Workshop on Commodity-Based Visualization Clusters}, October 27, 2002.
\end{enumerate}


\subsection*{Panels}
% DO NOT EDIT THIS FILE!
% This content is automatically generated from panel.bib.

\begin{enumerate}[label={[\arabic*]}, left=0pt]
\item  % ECP2023
  Lessons Learned from Porting Codes to GPUs.
  \emph{ECP Annual Meeting}, January 19, 2023.
\item  % LDAV2020
  Technical Advances in the Era of Big Data Analysis and Visualization: Large-Scale Visualization on Exascale Hardware.
  \emph{Large Data Analysis and Visualization (LDAV)}, October 25, 2020.
\item  % EGPGV2020
  How Ubiquitous Parallel Devices Affect Visualization.
  \emph{Eurographics Symposium on Parallel Graphics and Visualization (EGPGV)}, May 25, 2020.
\item  % DOECGF2018
  10-year Prognostication and How Do We Get There?.
  \emph{DOE Computer Graphics Forum}, April 25, 2018.
\item  % Vis2015
  Color Mapping in Vis: Perspectives on Optimal Solutions.
  \emph{IEEE Visualization}, October 2015.
\end{enumerate}


\section*{Professional Activities}
% DO NOT EDIT THIS FILE
% This content is automatically generated from
% /Users/4d5/web/kennethmoreland-com.github.io/curriculum_vitae/professional-activities.yaml'

\begin{description}
\item[Professional Organizations]~
  \begin{itemize}
  \item
    Institute of Electrical and Electronic Engineers (IEEE): Senior Member since 2023, Member since 1995.
  \item
    Association for Computing Machinery (ACM): Member since 1998.
  \end{itemize}
\item[Editor]~
  \begin{itemize}
  \item
    Associate Editor, IEEE Transactions on Visualization and Computer Graphics (TVCG): 2022--Present.
  \item
    Guest Editor, Parallel Computing: 2019.
  \end{itemize}
\item[Steering Committee Member]~
  \begin{itemize}
  \item
    IEEE Symposium on Large Data Analysis and Visualization (LDAV): 2020, 2021, 2022, 2023, 2024, 2025.
  \item
    In Situ Infrastructures for Enabling Extreme-scale Analysis and Visualization (ISAV): 2020, 2021, 2022, 2023, 2024, 2025.
  \item
    Workshop on In Situ Visualization (WOIV): 2018, 2021, 2022, 2023, 2024, 2025.
  \item
    International Symposium on Visual Computing (ISVC): 2020, 2021, 2022.
  \end{itemize}
\item[Event Chair/Co-Chair]~
  \begin{itemize}
  \item
    IEEE Symposium on Large Data Analysis and Visualization (LDAV): 2018, 2019.
  \item
    In Situ Infrastructures for Enabling Extreme-scale Analysis and Visualization (ISAV): 2019.
  \item
    Scientific Visualization \& Data Analytics Showcase, Supercomputing: 2018.
  \item
    VisLies: 2013, 2015, 2016, 2017, 2018, 2019, 2020, 2021, 2022, 2023, 2024.
  \end{itemize}
\item[Program Chair/Co-Chair]~
  \begin{itemize}
  \item
    IEEE Symposium on Large Data Analysis and Visualization (LDAV): 2016, 2017.
  \item
    Eurographics Symposium on Parallel Graphics and Visualization (EGPGV): 2013.
  \item
    In Situ Infrastructures for Enabling Extreme-scale Analysis and Visualization (ISAV): 2017, 2018.
  \end{itemize}
\item[Program Committee Member]~
  \begin{itemize}
  \item
    IEEE VIS: 2023.
  \item
    EuroVis: 2016, 2017, 2018, 2021, 2022, 2023.
  \item
    Scientific Visualization \& Data Analytics Showcase, Supercomputing: 2017, 2023.
  \item
    International Conference on Information Visualization Theory and Applications (IVAPP): 2022, 2023.
  \item
    International Symposium on Visual Computing (ISVC): 2013, 2014, 2019, 2020, 2021, 2022, 2023.
  \item
    IEEE VIS Short Papers: 2022.
  \item
    IEEE Scientific Visualization: 2014, 2015, 2016, 2017, 2019, 2020, 2021.
  \item
    Eurographics Symposium on Parallel Graphics and Visualization (EGPGV): 2009, 2010, 2011, 2012, 2013, 2015, 2016, 2017, 2019, 2020, 2021.
  \item
    IEEE Cluster: 2015, 2016, 2021.
  \item
    IEEE TPDS special section on Innovative R\&D toward the Exascale Era: 2021.
  \item
    VisGap: 2020.
  \item
    Symposium on Visualization: 2017.
  \item
    Visualization in High Performance Computing: 2015.
  \item
    Large Data Analysis and Visualization (LDAV): 2011, 2013, 2014.
  \item
    Ultrascale Visualization: 2013, 2014.
  \item
    International Conference for High Performance Computing, Networking, Storage, and Analysis (Supercomputing): 2012, 2013.
  \item
    Big Data Analytics: Challenges and Opportunities: 2012, 2013.
  \item
    Visualization Infrastructure \& Systems Technology: 2014.
  \end{itemize}
\item[Review Panels]~
  \begin{itemize}
  \item
    DOE SBIR Phase I: 2010, 2011, 2023.
  \item
    NSF IIS Division: 2012, 2014, 2015, 2016, 2018.
  \item
    DOE SBIR Phase II: 2012, 2014.
  \item
    DOE ECRP: 2009, 2013.
  \item
    DOE ASCR: 2009, 2010, 2011, 2013.
  \end{itemize}
\item[Miscellaneous]~
  \begin{itemize}
  \item
    Best Reviewer Award: EuroVis 2022.
  \item
    VIS Doctoral Colloquium Panelist: 2017.
  \end{itemize}
\end{description}


\section*{Software Artifacts}

The following is selected software I contributed to over (either as a developer or a project manager) the course of my career.

\begin{description}
\item[ParaView] A parallel scientific visualization application.
\item[Catalyst] An \emph{in situ} visualization library built on top of the ParaView application.
\item[VTK] A visualization library on which many tools, including ParaView, are built on top of.
\item[Viskores] A library of visualization algorithms designed to run on multi- and many-core processors such as GPUs.
\item[IceT] A parallel rendering library.
\item[VTK-m] The predecessor of Viskores.
\item[Dax] A predecessor of VTK-m.
\end{description}

\end{document}
